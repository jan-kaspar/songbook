\transposition{-3}

Předkové [Gmi]nelhali. Už je to [Dmi]tak.
To když nám [F]říkali: Každýmu jednou [Gmi]vlak
dojede do cí[Eb]le a poslední mí[Cmi]le je smutný finá[Gmi]le. [D]

Jsme stromy s [Gmi]osudem a vzácnej [Dmi]druh.
Každý rok [F]přibude nám další leto[Gmi]kruh.
Nabydem [Eb]objemu svou silou [Cmi]z kořenů
a potom [Gmi]zůstane jenom prach a [D]vzduch.

Víš, tady [Gmi]neplatí nejlepší [Dmi]nakonec.
Cesta se [F]klikatí, najednou [Gmi]spadla klec.
Pak můžeš [Eb]děkovat, svědomí [Cmi]zpytovat,
nečekej [B]nepřijde životní sluno[F]vrat.

\ref
[B]Né, ještě [Gmi]né, ještě [Dmi]není doba [F]zlá.
[Cmi]Už se kácí v [Eb]našem lese, [B]zdrhej kam se [F]dá.
[B]Dej, ještě [Gmi]dej, ještě [Dmi]prosím aspoň [F]den.
[Cmi]Dochází nám [Eb]letokruhy, [B]zítra nebu[F]dem.

Co v mládí získáš, stáří ukradne.
A později či dřív každej strom upadne.
Za krásnou vteřinu platíme hodinu ze svýho finále.

Máme to v dlaních svých, v osudu napsaný.
A v dlouhých příbězích nedá se gumovat.
Možná tak litovat, zoufat a bědovat.
Nečekej, nepříjde životní slunovrat.

\ref

\ref

Předkové nelhali...
