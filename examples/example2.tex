\input songbook

%----------------------------------------------------------------------------------------------------

\bsong{A caminhada}{Alberto Pinto 390-Mafamude}

% song data here
% A  B   C    D   E   F   G
% La Si  Do   re  Mi  Fa  Sol

% this will make the transposition engine use Do, Re, ...
\FrenchChordStyle

% to enable the transposition engine 
\transposition{0}

% There is however a tiny issue: if you enter Eb, it will output Re#, not Mib.
% In fact, after a general transposition, the engine cannot know
% whether you would have preferred to use b or #, so it consistently
% uses just sharps...

\annot{intro:}
\bextra
			C (0, 3, 0, 8)
			
            Am (0, 3, 0, 8)
            
            F (0, 3, 0, 8)
            
            G (0, 3, 0, 8)
 
% this new macro does what you need
% it is easy to change the font, the macro definition is at the end of songbook.tex          
\comment{tab in the end of the page} 
\eextra
		       

[C]Vem viver a amizade, [Am]que jamais tem idade
[F]Construir o Mundo [G]à Tua vontade.
[C]Partilhar como irmãos [Am]esta nossa canção                               
[F]Para sempre unidos [G]o Escutismo ajudar.

\ref
[C]Tu és jovem e jovem hás-de [Am]ser
Se queres permane[F]cer nesta u[G]nião [F]de ami[G]zade.
[C]Tu és jovem, tens força para lu[Am]tar
Tens a garra para [F]amar
[G]És Caminheiro,[F]\ do mundo [G]inteiro\ \ [F]
Serás o [G]primeiro a che[C]gar. \comment{bis}



\bextra
% should be in italic or some way different from the refrain. 
% some separation from the refrain would be wellcomed


\comment{repeat intro}

\eextra

[C]Enfrentar a caminhada [Am]que a Vida há-de propor
[F]Construir o Mundo [G]com o seu grande valor
[C]Liberdade de um Homem, [Am]onde a alegria chegou
[F]E a força de gritar, [G]dum alerta que mudou.



\annot{intro tab:}

\bextra
\comment{to be played in the beginning, and every time after the refrain:}
\eextra
\btab
E -0---0----3----0----8----0---0----3----0----8----
B -1-----------------------1-----------------------
G -0-----------------------2-----------------------
D -2-----------------------2-----------------------
A -3-----------------------0-----------------------
E -------------------------------------------------
\etab

\eject

\btab
E -1---0----3----0----8----3---0----3----0----8----
B -1-----------------------3-----------------------
G -2-----------------------0-----------------------
D -3-----------------------0-----------------------
A -3-----------------------2-----------------------
E -1-----------------------3-----------------------
\etab

\esong


\bye




