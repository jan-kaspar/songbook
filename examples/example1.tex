\input ../processing/songbook

% you can adjust the bottom of each page
% the following line would make:
%     left: date
%     middle: name of current song
%     right: page number
\footline = {{\ChordItalicsFont \number\day. \number\month. \number\year\hfil \currentLabel \hfil\the\pageno}}

% the following line makes
%     middle: page number
\footline = {{\hfil\the\pageno\hfil}}

%----------------------------------------------------------------------------------------------------

\bsong{Some title here}{Group name}

% song data here

% line-ends are respected
There she stood in the doorway;  I heard the mission bell
And I was thinking to myself 
This could be heaven or this could be hell

% an empty line starts a new verse
Then she lit up a candle, and she showed me the way
There were voices down the corridor, 
I thought I heard them say...

% verses are numbered automatically, if you want a refrain
\ref
There she stood in the doorway;  I heard the mission bell
And I was thinking to myself 
This could be heaven or this could be hell

% if you want a custom text:
\annot{custom}
Then she lit up a candle, and she showed me the way
There were voices down the corridor, 
I thought I heard them say...

% chords are entered in between [...] and will be aligned above the following syllabus
[C]Time can [G]bring you [Ami]down,
time can [D]bend your [G]knees

% you can use CDEFGAB to enter chords, as well as # and b for sharp and flat
[C#]Time can [Ab]bring you [A#mi]down

% it doesn't matter what follows the tone (e.g. m, mi, maj, ...)
% special base notes can be entered after "/"
[C3-4]Time can [Gmaj]bring you [Ami/C]down

% use "\ " for force spaces, e.g. lyrics starts later than a chord
[Am]\ On a dark desert highway, [E7]\ cool wind in my hair

% you and include a block typeset into two columns
\bcol
% 1st column data here
[C]Time can [G]bring you [Ami]down,
time can [D]bend your [G]knees[D, Emi, D, G]
[C]Time can [G]break your [Ami]heart,
have you [D]begging [G]please [D]begging [E]please
\ncol
% 2nd column data here
[C]Time can [G]bring you [Ami]down,
time can [D]bend your [G]knees[D, Emi, D, G]
[C]Time can [G]break your [Ami]heart,
have you [D]begging [G]please [D]begging [E]please
\ecol

% you can force page end (if you skip the \vfil, the page contents will be streched to occupy full page)
\vfil\eject

% you can have your chords transposed, enter the shift in half-tones
\transposition{+2}

[C]Time can [G]bring you [Ami]down,
time can [D]bend your [G]knees[D, Emi, D, G]
[C]Time can [G]break your [Ami]heart,
have you [D]begging [G]please [D]begging [E]please

% to cancel the trasposition, set the shift to 0
\transposition{0}

[C]Time can [G]bring you [Ami]down,
time can [D]bend your [G]knees[D, Emi, D, G]
[C]Time can [G]break your [Ami]heart,
have you [D]begging [G]please [D]begging [E]please

% this is the way to enter text in the standard TeX formatting
\bextra
There is something
written here.
And what comes below is a tab:
\eextra

% you can use tabs too
\btab
e -----------------0----------------------
h --------------0-------------------------
g -----------1----------------------------
d --------2-------------------------------
A -----2----------------------------------
E --0-------------------------------------
\etab

% finally, there are symbols for repetition (the "{}" is needed to preserve the space after a command)
before \brep repeated text \erep{} after

\esong


%----------------------------------------------------------------------------------------------------

% you can include a song from an external file
\input example1_included

%----------------------------------------------------------------------------------------------------

% insert table of contents

\centerline{\TitleFont Table of contents}
\vskip5mm

\InsertTOC


% stop TeX processing
\bye
