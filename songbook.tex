\iffalse
------------ TODO --------------
* empty line after \etab
\fi

\input colors

%----------------------------------------------------------------------------------------------------
%-------------------- fonts, struts
%----------------------------------------------------------------------------------------------------

\font\TitleFont = cs-qplb-sc at 25pt
\font\AuthorFont = cs-qplr-sc at 18pt
\font\InitialFont = cs-qplb-sc at 40pt
%\font\InitialFont = v_CurlzMT_cz at 40pt
\font\FootlineFont = cs-qplri at 12pt

\def\TextFontName{cs-qplr }
\def\RefFontName{cs-qplr }
\def\AnnotationFontName{cs-qplb }

\def\NormalFonts{%
	\font\TextFont = \TextFontName at 12pt
	\font\RefFont = \RefFontName at 12pt
	\font\ChordFont = cs-qplr at 12pt
	\font\ChordItalicsFont = cs-qplri at 12pt
	\font\ChordScriptFont = cs-qplr at 9pt
	\font\AnnotationFont = \AnnotationFontName at 12pt
	% TODO
	\font\TabFont = t1-zi4r-7 at 12pt
	%\font\TabFont = ec-inconsolata at 12pt
	\font\mathSymbFont = cmmi12 at 12pt
	\textfont1=\mathSymbFont
	%
	\baselineskip14pt
	%
	\setbox\strutbox=\hbox{\vrule height10.2pt depth4.2pt width0pt}%
	%
	\def\ChordStrut{\vrule width0pt height9pt depth3pt}%
}

\def\SmallerFonts{%
	\font\TextFont = \TextFontName at 10pt
	\font\ChordFont = cs-qplr at 10pt
	\font\ChordItalicsFont = cs-qplri at 10pt
	\font\ChordScriptFont = cs-qplr at 7.5pt
	\font\AnnotationFont = cs-qplb at 10pt
	% TODO
	\font\TabFont = cs-qplr at 10pt
	%\font\TabFont = ec-inconsolata at 10pt
	\font\mathSymbFont = cmmi12 at 10pt
	\textfont1=\mathSymbFont
	%
	\baselineskip12pt
	%
	\setbox\strutbox=\hbox{\vrule height8.5pt depth3.5pt width0pt}%
	%
	\def\ChordStrut{\vrule width0pt height8pt depth2pt}%
}


%----------------------------------------------------------------------------------------------------
%-------------------- layout, dimensions, skips, etc.
%----------------------------------------------------------------------------------------------------

\vsize=297mm
\hsize=210mm

% the actual space between text and page edges
\newdimen\hmargin \hmargin=25mm
\newdimen\vmargin \vmargin=15mm

\hoffset=\hmargin \advance\hsize-2\hoffset \advance\hoffset-1in
\voffset=\vmargin \advance\vsize-2\voffset \advance\voffset-1in

% the horizontal unprintable margin
\newdimen\PrinterMargin
\PrinterMargin6mm

\parindent0pt
\lineskip0pt
\lineskiplimit0pt

%\footline = {\hfil\ChordScriptFont typeset with kaspi\TeX}
\footline = {{\FootlineFont \number\day. \number\month. \number\year\hfil \currentLabel \hfil\the\pageno}}

\newskip\VerseSkip
\VerseSkip=\baselineskip

\newskip\TitleSkip
\TitleSkip=2\baselineskip

\newskip\InterChordSkip
\InterChordSkip=1mm

\newdimen\ColumnSeparation
\ColumnSeparation=1cm

%----------------------------------------------------------------------------------------------------
%-------------------- colors
%----------------------------------------------------------------------------------------------------

\def\TextColor{\cBlack}
\def\ChordColor{\cRed}
\def\BassColor{\cBlue}
\def\InitialBackgroundColor{\cmyk{0 0 0 0.4}}%
\def\InitialForegroundColor{\cmyk{0 0 0 0}}%


% # will be used for sharps, @ will replace it
\catcode`\@=6
\catcode`\#=11

%----------------------------------------------------------------------------------------------------
%-------------------- chord macros
%----------------------------------------------------------------------------------------------------

\newif\ifDoTransposition
\newcount\TranspositionShift

\def\transposition@1{%
	\DoTranspositiontrue
	\TranspositionShift@1
}

\def\FirstTone{C}
\def\SecondTone{D}
\def\ThirdTone{E}
\def\FourthTone{F}
\def\FifthTone{G}
\def\SixthTone{A}
\def\SeventhMinusTone{B}
\def\SeventhTone{H}

\def\CzechChordStyle{%
	\def\SeventhMinusTone{B}%
	\def\SeventhTone{H}%
}

\def\EnglishChordStyle{%
	\def\SeventhMinusTone{}%
	\def\SeventhTone{B}%
}

\def\DefaultChordStyle{}

\newcount\ChordCode

\def\ToneCodeToLabel{%
	\ifcase\ChordCode%
			C%
		\or C$\sharp$%
		\or D%
		\or D$\sharp$%
		\or E%
		\or F%
		\or F$\sharp$%
		\or G%
		\or G$\sharp$%
		\or A%
		\or A$\sharp$%
		\or\SeventhTone%
	\fi
}

\def\FrenchChordStyle{%
	\def\ToneCodeToLabel{%
		\ifcase\ChordCode%
				Do%
			\or Do$\sharp$%
			\or Re%
			\or Re$\sharp$%
			\or Mi%
			\or Fa%
			\or Fa$\sharp$%
			\or Sol%
			\or Sol$\sharp$%
			\or La%
			\or La$\sharp$%
			\or Si%
		\fi
	}
}

\def\RelativeChordStyle{%
	\def\ToneCodeToLabel{%
		\ifcase\ChordCode%
				I%
			\or I$\sharp$%
			\or II%
			\or II$\sharp$%
			\or III%
			\or IV%
			\or IV$\sharp$%
			\or V%
			\or V$\sharp$%
			\or VI%
			\or VI$\sharp$%
			\or VII%
		\fi
	}
}

\def\ToneLabelToCode@1{%
	\edef\temp{@1}%
	\ifx\temp\FirstTone\ChordCode=0\fi
	\ifx\temp\SecondTone\ChordCode=2\fi
	\ifx\temp\ThirdTone\ChordCode4\fi
	\ifx\temp\FourthTone\ChordCode5\fi
	\ifx\temp\FifthTone\ChordCode7\fi
	\ifx\temp\SixthTone\ChordCode9\fi
	\ifx\temp\SeventhMinusTone\ChordCode10\fi
	\ifx\temp\SeventhTone\ChordCode11\fi
}

\def\MakeChordBuffer@1{%
	\def\ChordText{@1}%
	\ifx\ChordText\empty
		\def\ChordBuffer{}%
	\else
		\def\ChordBuffer{\ChordColor}%
		\ProcessChordText
		\edef\ChordBuffer{\ChordBuffer\TextColor}%
	\fi
}

\def\ChordGetFirstChar@1@2\end{%
	\def\ChordFirstChar{@1}%
	\def\ChordText{@2}%
}

\def\ProcessChordText{%
	\def\bChar{b}%
	\def\sChar{#}%
	\def\slChar{/}%
	\def\cmChar{,}%
	\def\mChar{m}%
	\def\iChar{i}%
	%
	\def\ChordTone{}%
	\def\ChordType{}%
	\def\ChordAlternations{}%
	%
	\if\ChordText\empty
	\else
		\expandafter\ChordGetFirstChar\ChordText\end%
		\edef\ChordTone{\ChordFirstChar}%
		\ToneLabelToCode\ChordFirstChar
		%
		\if\ChordText\empty
		\else
			\expandafter\ChordGetFirstChar\ChordText\end%
			\ifx\ChordFirstChar\sChar
				\advance\ChordCode+1
				\edef\ChordTone{\ChordTone$\sharp$}%
			\else
				\ifx\ChordFirstChar\bChar
					\advance\ChordCode-1
					\edef\ChordTone{\ChordTone$\flat$}%
				\else
					\edef\ChordText{\ChordFirstChar\ChordText}%
				\fi
			\fi
		\fi
		%
		\ifDoTransposition
			\advance\ChordCode\TranspositionShift
			\ifnum\ChordCode>11
				\advance\ChordCode-12
			\fi
			\ifnum\ChordCode<0
				\advance\ChordCode+12
			\fi
			\edef\ChordTone{\ToneCodeToLabel}%
		\fi
	\fi
	%
	\ProcessChordType
	%
	\ChordFlushToBuffer
}

\def\ProcessChordType{%
	\ifx\ChordText\empty
	\else
		\expandafter\ChordGetFirstChar\ChordText\end%
		%
		\ifx\ChordFirstChar\slChar
			\ChordFlushToBuffer
			\edef\ChordBuffer{\ChordBuffer\TextColor/\BassColor}%
			\ProcessChordText	
		\else
			\ifx\ChordFirstChar\cmChar
				\ChordFlushToBuffer
				\edef\ChordBuffer{\ChordBuffer\hskip\InterChordSkip\ChordColor}%
				\ProcessChordText	
			\else
				\ifcat\ChordFirstChar1%
					\edef\ChordAlternations{\ChordAlternations\ChordFirstChar}%
				\else
					\if\ChordFirstChar\mChar
						\edef\ChordType{\ChordType m}%
						\ifx\ChordText\empty
						\else
							\expandafter\ChordGetFirstChar\ChordText\end%
							\ifx\ChordFirstChar\iChar
							\else
								\edef\ChordText{\ChordFirstChar\ChordText}%
							\fi
						\fi
					\else
						\edef\ChordType{\ChordType\ChordFirstChar}%
					\fi
				\fi
			\fi
		\fi
		%
		\ProcessChordType
	\fi
}

\def\ChordFlushToBuffer{%
	\ifx\ChordTone\empty
	\else
		\edef\ChordBuffer{%
			\ChordBuffer
			{\ChordFont\ChordTone}%
			{\ChordItalicsFont\ChordType}%
			\raise3pt\hbox{\ChordScriptFont\ChordAlternations}%
		}%
		\def\ChordTone{}%
		\def\ChordType{}%
		\def\ChordAlternations{}%
	\fi
}

\def\NoChords{%
	\def\MakeChordBuffer@@1{}%
}

%----------------------------------------------------------------------------------------------------
%-------------------- verses and refrains
%----------------------------------------------------------------------------------------------------

\newcount\VerseCounter
\newif\ifAnnotatedVerse
\newif\ifPrintAnnotation
\def\VerseAnnotation{}

\def\verse{%
	\AnnotatedVersefalse
}

\def\annotate@1{%
	\AnnotatedVersetrue
	\xdef\VerseAnnotation{@1}%
	\TextFont
}

\let\annot=\annotate

\def\ref{%
	\annotate{R:}%
	\RefFont
}

\def\plain{%
	\annotate{}%
}

\def\lside@1{%
	\noindent
	\llap{\AnnotationFont@1\hskip2mm}%
}

\def\AddVerseAnnotation{%
	\ifPrintAnnotation
		\ifAnnotatedVerse
			\lside{\VerseAnnotation}%
		\else
			\global\advance\VerseCounter by 1
			\lside{\the\VerseCounter .}%
			\TextFont
		\fi
		\verse
		\PrintAnnotationfalse
	\fi
}

%----------------------------------------------------------------------------------------------------
%-------------------- table of contents
%----------------------------------------------------------------------------------------------------

\def\tocBuffer{}

\def\MakeTOCLine@1@2{%
	\line{\strut@1\ \leaders\hbox{.\ } \hfil\ @2}
}

\def\InsertTOC{%
	\tocBuffer
}

%----------------------------------------------------------------------------------------------------
%-------------------- titles
%----------------------------------------------------------------------------------------------------

\def\Nothing@1{%
	\setbox0\hbox{@1}%
}

\def\SetInitial@1{%
	\gdef\Initial{@1}%
}

\def\WithBlackBackground@1{%
	\bgroup
		\advance\PrinterMargin+50pt
		\InitialBackgroundColor%
		\rlap{\vrule height37pt width\PrinterMargin depth13pt}%
	\egroup
	\InitialForegroundColor%
	\hbox to50pt{\hss@1\hss}%
	\hskip\PrinterMargin
	\TextColor%
}

% PDF bookmark counter
\newcount\pdfMarkNumber
\pdfMarkNumber=0

\def\currentLabel{}

\def\MakeTitle@1@2{%
	%
	\def\temp{@2}%
	\ifx\temp\empty
		\xdef\currentLabel{@1}
	\else
		\xdef\currentLabel{@1 (@2)}
	\fi
	%
	\global\advance\pdfMarkNumber by1
	\pdfdest num \pdfMarkNumber xyz
	\xdef\pdfChapterMarkNum{\the\pdfMarkNumber}%	
	\pdfoutline goto num \pdfChapterMarkNum count 0 {@1 (@2)}%
	%
	\xdef\tocBuffer{\tocBuffer\MakeTOCLine{\currentLabel}{\the\pageno}}
	%
	\Nothing{\SetInitial@1}%
	\line{%
		\hss
		\TitleFont@1%
		\hss
		\rlap{%
			\hbox to\hmargin{%
				\hss
				\vbox to0pt{%
					\vss
					\hbox{\WithBlackBackground{\InitialFont\Initial}}%
					\vss
				}%
			}%
		}%
	}%
	%
	\vskip2mm
	%
	\def\temp{@2}%
	\ifx\temp\empty
		\vskip\baselineskip
	\else
		\line{\AuthorFont\hss@2\hss}%
	\fi
	%
	\vskip\TitleSkip
}

%----------------------------------------------------------------------------------------------------
%-------------------- song data parsing
%----------------------------------------------------------------------------------------------------

\bgroup
\catcode`\^^M=\active
\gdef\bsong@1@2{
	\begingroup
	\MakeTitle{@1}{@2}%
	%
	\global\VerseCounter = 0
	%
	\verse
	\PrintAnnotationtrue
	\AnnotatedVersefalse
	\ForceVerseSeparatorfalse
	\DoTranspositionfalse
	\DefaultChordStyle
	%
	\NormalFonts
	\TextFont
	\TextColor%
	%
	\def\ {\hskip1em}
	%
	\everypar={\HandleLine}%
	\catcode`\^^M=\active%
	\gdef\HandleLine@@1^^M{\ProcessLine{@@1}\TextLineEnd}%
	\let^^M=\OtherLineEnd%
}
\egroup

\def\esong{%
	\everypar={}%
	\ifAnnotatedVerse
		\ProcessLine{}%
	\fi
	\endgroup
	\vfil\eject
	\ifodd\pageno
	\else
		\bgroup
			\footline={}
			\hbox{}
			\vfil\eject
		\egroup
	\fi	
	
	\xdef\currentLabel{}
}

\def\ProcessLine@1{%
	\def\ChordBuffer{}%
	\AddVerseAnnotation
	%\vrule
	\SubProcessLine@1[]\end
	\def\VerseAnnotation{}%
}

\def\SubProcessLine@1[@2]@3\end{%
	\vbox{%
		\offinterlineskip
		\ifx\ChordBuffer\empty
			%\hbox{}%
		\else
			\hbox{\ChordColor\ChordStrut\ChordBuffer\hskip\InterChordSkip}%
		\fi
		\def\temp{@1}%
		\hbox{\strut@1}%
	}%
	%
	\MakeChordBuffer{@2}%
	%
	\def\temp{@3}%
	\ifx\temp\empty
		\def\next{}%
	\else
		\def\next{\SubProcessLine@3\end}%
	\fi
	\next
}

\newif\ifForceVerseSeparator

\def\TextLineEnd@1{%
	\endgraf
	\ifx@1\OtherLineEnd
		\VerseSeparator
	\else
		\ifForceVerseSeparator
			\VerseSeparator
			\ForceVerseSeparatorfalse
		\fi
	\fi
	@1%
}

\def\OtherLineEnd@1{%
	\ifx@1\OtherLineEnd
		\ifAnnotatedVerse
			\ForceVerseSeparatortrue
			\leavevmode
		\fi
	\fi
	@1%
}

\def\VerseSeparator{%
	%\hrule
	\vskip\VerseSkip
	%\hrule
	\PrintAnnotationtrue
}

%----------------------------------------------------------------------------------------------------
%-------------------- two-column typesetting
%----------------------------------------------------------------------------------------------------

\def\bcol{%
	\hbox\bgroup%
		\vtop\bgroup
			\advance\hsize-\ColumnSeparation\divide\hsize2
}

\def\ncol{%
		\egroup
		\hskip\ColumnSeparation%
		\vtop\bgroup
			\advance\hsize-\ColumnSeparation\divide\hsize2
}

\def\ecol{%
		\egroup
	\egroup
}

%----------------------------------------------------------------------------------------------------
%-------------------- extra/non-formatted sections
%----------------------------------------------------------------------------------------------------

\def\ExtraEveryPar{%
	\ifAnnotatedVerse
		\lside{\VerseAnnotation}%
	\fi
	\everypar={}%
}

\def\noobeylines{%
	\catcode`\^^M=5
}

\def\bextra{%
	\bgroup%
	\noobeylines%
	\everypar={\ExtraEveryPar}%
}

\def\eextra{%
	\egroup
	\par
	\VerseSeparator
	\verse
}

%----------------------------------------------------------------------------------------------------
%-------------------- tabulatures
%----------------------------------------------------------------------------------------------------

\def\ObeyLines{%
	\def\par{%
		\endgraf
		\leavevmode
	}%
	\obeylines
}

\def\btab{%
	\bgroup%
		\everypar={\ExtraEveryPar}%
		\TabFont
		\def\par{\endgraf}%
		\ObeyLines
		\obeyspaces
}

\def\etab{%
	\unskip
	\egroup
	\VerseSeparator
	\verse
}

%----------------------------------------------------------------------------------------------------
%-------------------- repetitions
%----------------------------------------------------------------------------------------------------

\def\brep{%
	\leavevmode
	%\hbox{\vrule height10pt depth2pt}%
	%\hskip0.5ex
	%\hbox{\vrule height10pt depth2pt}%
	%\hskip0.5ex
	\char91:
}

\def\erep{%
	\hskip0.5ex
	:\char93
	%\hbox{\vrule height10pt depth2pt}%
	%\hskip0.5ex
	%\hbox{\vrule height10pt depth2pt}%
}

\def\comment@1{{\ChordItalicsFont@1}}

%----------------------------------------------------------------------------------------------------
%-------------------- initial settings
%----------------------------------------------------------------------------------------------------

\NormalFonts
\TextFont
\TextColor
